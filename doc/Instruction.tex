\documentclass{article}
\usepackage[utf8]{inputenc}
\usepackage{hyperref}
\usepackage{color}
\usepackage{tikzsymbols}

\title{Deep Learning - Project II}
\author{TA: Aditya, Anirban, Lokesh}
\date{February 02, 2020}

\begin{document}

\maketitle

\section*{Instructions}

\begin{itemize}
    \item This is a coding assignment. The code has to be written in Python. You have to use \textbf{Python $>=$ 3}, \textbf{pytorch $>=$ 1.2} and \textbf{tensorflow $>=$ 2.0} to solve all the problems. Please mention the version you have used in the report.
    
    \item The assignment deliverable is your python code, saved model and your report of observations.
    
    \item A google form will be sent through Piazza. You need to fill the form with your github repository details.
        
    \item You are encouraged to discuss among yourselves, but \textbf{DO NOT COPY} solutions or code. The consequences will be severe.
    
    \item Read the complete assignment carefully, before attempting to solve it.
    
    \item Follow the instructions provided for test execution of your code, the formats for input and output files and the naming conventions. \hyperlink{instruction}{ \color{blue}\footnotesize | Detailed Instructions}
    
    \item The submission deadline is \textbf{Feb 19, 2020}.
\end{itemize}

\quad \quad{\framebox{\textbf{Best of luck for the assignment. HAPPY CODING!}}}


\section*{\hypertarget{instruction}How to run}
 You should use the testdata given in the API (pytorch/tensorflow) to test your code and output the answers to files \texttt{multi-layer-net.txt} and \texttt{convolution-neural-net.txt}.

\noindent To test the model we run the code :\\
\texttt{python main.py}\\

Note that you should use the saved models when obtaining the output on the test data.

\section*{Assignment Deliverables and Evaluation of Project Report}

\begin{itemize}
    
    \item Prepare a report and name it \texttt{Deep\_Learning\_Report\_2.pdf}. In your report, you need to briefly describe what you have done, present the results (in a form you think is good) and provide a brief discussion of the results you obtained. Please feel free to play with the model architecture and hyper-parameter settings. Please make it concise and to-the-point.

    \item The name of output file for various tasks needs to be \textbf{multi-layer-net.txt} and \textbf{convolution-neural-net.txt}. Put them into the same directory of \textit{main.py} file.
    
    \item You need to save the trained model in a specific format and put it in \textit{model} directory. During test time, your code should use this saved model and not retrain it from scratch.
    
    \item Designing machine learning models, especially deep learning models is more of an art than a science. Hence, we would evaulate your report based on your answers to the following questions -
\begin{itemize}
\item What made you choose your current architecture?
\item What guided your choice of hyperparameters - number of neurons/layers, activation function, learning rate etc.?
\item How did you validate your model? - Note that validating the model is different from testing it!!
\end{itemize}

There are no right answers to these questions and the answers are based on judgement. Higher credit will be given to answers which are more logically/rationally grounded. For example, the reasoning:

\begin{quote}
I have chosen ResNet because it allows having large number of layers without problems such as vanishing gradients, etc.  
\end{quote}

is better than

\begin{quote}
I have chosen this architecture because $XYZ$ used it.
\end{quote}
\end{itemize}


\section*{Starter Code}

A starter code will be provided. The output files should have the same format as the one provided in this repo.

\end{document}